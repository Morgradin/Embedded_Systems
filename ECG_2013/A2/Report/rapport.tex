\documentclass[12pt,a4paper]{article}
\usepackage{ucs}
\usepackage{caption}
\usepackage[latin1,utf8x]{inputenc}
\usepackage{amsmath}
\usepackage{caption}
\captionsetup{font=small,labelfont=bf}
\usepackage[danish]{babel}
\usepackage[rmargin=3cm,tmargin=3.3cm]{geometry}
\usepackage{listings}
\usepackage{color}
\setlength{\parindent}{0pt}
\setlength{\parskip}{1ex plus 0.5ex minus 0.2ex}
\usepackage{graphicx}
\usepackage{fixltx2e}


%insert links
\usepackage{hyperref}
\usepackage{fancyhdr,lastpage}	
\pagestyle{fancy}


\definecolor{mygreen}{rgb}{0,0.6,0}
\definecolor{myblue}{rgb}{0,0,1}
\definecolor{myyellow}{rgb}{0.7,0.7,0}
\definecolor{myblack}{rgb}{0,0,0}

\lstset{
	breaklines=true,
	numbers=left, 
	commentstyle=\color{mygreen},
	stringstyle=\color{myyellow},
}

%header
\lhead{ 
	Embedded Systems \\
	02131 \\ 
}
\chead{ 
}
\rhead{ 2 October, 2012 \\ \bigskip  }

%Footer
\lfoot{
	\rule{\textwidth}{0.1mm}\\
}

\cfoot{}
\rfoot{\ \\ \scriptsize{Side \thepage\ af \pageref{LastPage}}}

\begin{document}

%Forside
\begin{titlepage}
	\begin{center}
		\vspace*{13\baselineskip}
		\huge
		\bfseries
		Embedded Systems\\ 
		\ \\
		02131 \\[5\baselineskip]

		\normalfont
		\Large
		R-peak detection!\\	
		2013

		\small
		\vfill
	\end{center}	
	\begin{flushleft}
		Jakob Welner, s124305\\
	 	Jacob Gjerstruo, s113440\\
	\end{flushleft}
\end{titlepage}

\ \\
\section*{Abstract}


\thispagestyle{empty} 
\newpage

%Table of Contents
\tableofcontents
\thispagestyle{empty} 
\newpage

%Reset pagecount
\setcounter{page}{1}

%Alm. sider
\ \\
\section{Introduction}
	After we succesfully proved in assignment 1 that the QRS algorithm could be implemented, Medembed has hired us to implement a small embedded processor. This processor needs to be able to execute the QRS algorithm on a proof-of-concept basis, done by implementing just one of the filters to demonstrate the performance of the processor.	
\subsection{Requirements}
Below follows a list of functional and non-functional requirements:\\

\textbf{ Functional requirements for the application:}
\begin{itemize}
	\item Each module of the processor must first be built, run and tested by their own
	\item An instruction set must be designed
	\item A controller for the instruction set must be designed
	\item The controller must be integrated into a larger system
	\item The performance of the system must be analysed in terms of speed and memory requirements

\end{itemize}
\textbf{Non-functional requirements for the application:}
\begin {itemize}
	\item The processor must be implemented with the use of Gezel, a Hardware Description Language (GHDL)
\end{itemize}

\section{Theory}
 	In order to initiate the structure- and design-process of the program, a number of questions needed to be answered first:\\
 	
 	\begin{enumerate}
	\item Which processor modules do we need and how should these be made?
	\item What instructions must the instruction set contain?
	\item How do we implement a controller that can understand and ``udf�re'' the instruction set?
	\item How do we integrate the controller into the system?
	\item What are the critical parts of the system and how can these be analysed?
\end{enumerate}

\subsection{Problem 1: Processor modules}
	
\subsection{Problem 2: Instruction set}
	
\subsection{Problem 3: Implementation of the controller}

\subsection{Problem 4: System Integration}

\subsection{Problem 5: Critical parts}

\section{Design}

\section{Implementation}

\subsection{Processor modules}

\subsection{Instruction set}
	
\subsection{Implementation of the controller}
	
\subsection{System Integration}
	
\subsection{Critical parts}

\section{Results}
	
%	\begin{figure}[h!]
%		\centering
%			\includegraphics[width=1\textwidth]{Screenshots/tests_filter_result.png}
%		\caption{A screenshot of the output of the tests of the filters.}
%		\label{test_filter_result}
%	\end{figure}

\section{Discussion}
	
\subsection{Improvements}

\section{Conclusion}
\newpage
\begin{thebibliography}{9}

\bibitem{lamport94}
  Michael Reibel Boesen, Jan Madsen, Linas Kaminskas, Paul Pop, Karsten Juul Frederiksen\\
  \emph{Assignment 2: The ECG processor}\\
  2013.\\

\bibitem{Gezel}
  \emph{Lecture7: Finite state machine with Datapath}\\
  Fall, 2007.\\

\bibitem{GezelBasicSyntax}
  \emph{GEZEL Basic Syntax}\\
\end{thebibliography}
	
\newpage	
	\begin{Large}
		\textbf{Appendix}
	\end{Large}
	\appendix

\section{Who wrote what}
Jacob Gjerstrup, s113440 wrote: \\
Jakob Welner, s124305 wrote: \\

\section{Output of the RPeakDetection}
	
\section{Sourcecode - introductionary exercises}
	
\section{Sourcecode - the real program}

\subsection{Buffer}
	%\lstinputlisting[language=C]{Code/buffer.c}	
\subsection{Filters}
\subsection{RPeakDetection}
\subsection{Sensor}
\subsection{Header files}
\subsubsection{sensor.h}
\subsubsection{buffer.h}
\subsection{Tests}
\end{document}