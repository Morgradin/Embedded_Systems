\documentclass[12pt,a4paper]{article}
\usepackage{ucs}
\usepackage{caption}
\usepackage[latin1,utf8x]{inputenc}
\usepackage{amsmath}
\usepackage{caption}
\captionsetup{font=small,labelfont=bf}
\usepackage[danish]{babel}
\usepackage[rmargin=3cm,tmargin=3.3cm]{geometry}
\usepackage{listings}
\usepackage{color}
\setlength{\parindent}{0pt}
\setlength{\parskip}{1ex plus 0.5ex minus 0.2ex}
\usepackage{graphicx}
\usepackage{fixltx2e}

\usepackage[T1]{fontenc}
\usepackage{textcomp}


%insert links
\usepackage{hyperref}
\usepackage{fancyhdr,lastpage}	
\pagestyle{fancy}


\definecolor{mygreen}{rgb}{0,0.6,0}
\definecolor{myblue}{rgb}{0,0,1}
\definecolor{myyellow}{rgb}{0.7,0.7,0}
\definecolor{myblack}{rgb}{0,0,0}

\lstset{
	basicstyle=\ttfamily\footnotesize
	breaklines=true,
	numbers=left, 
	commentstyle=\color{mygreen},
	stringstyle=\color{myyellow},
}

%header
\lhead{ 
	Embedded Systems A2\\
	02131 \\ 
}
\chead{ 
}
\rhead{ 11 November, 2013 \\ \bigskip  }

%Footer
\lfoot{
	\rule{\textwidth}{0.1mm}\\
}

\cfoot{}
\rfoot{\ \\ \scriptsize{Side \thepage\ af \pageref{LastPage}}}

\begin{document}

%Forside
\begin{titlepage}
	\begin{center}
		\vspace*{13\baselineskip}
		\huge
		\bfseries
		Embedded Systems\\ 
		\ \\
		02131 \\[5\baselineskip]

		\normalfont
		\Large
		R-peak detection. \\
		Assignment 2\\	
		2013

		\small
		\vfill
	\end{center}	
	\begin{flushleft}
		Jakob Welner, s124305\\
	 	Jacob Gjerstrup, s113440\\
	\end{flushleft}
\end{titlepage}

\ \\
\section*{Abstract}
The task of this assignment was to show by proof-of-concept that a dedicated processor could be created through the Gezel hardware description language. This dedicated processor was created in such a way that it could accept instructions in the form of assembly commands, enabling it to execute the Moving Window Integration filter from Assignment 1. The dedicated processor was then attached to a bus, which again was attached to an external memory module where the data samples were located. A protocol for communicating with the bus and memory module was then implemented.

\thispagestyle{empty} 
\newpage

%Table of Contents
\tableofcontents
\thispagestyle{empty} 
\newpage

%Reset pagecount
\setcounter{page}{1}

%Alm. sider
\ \\
\section{Introduction}
	After successfully proving in Assignment 1 that the QRS algorithm could be implemented in software, Medembed now wants a proof of concept that the algorithm can be optimized further by implementing the most demanding parts of it in hardware. From Assignment 1 it was found that the Moving Window Implementation was taking up the most processing power so it was decided to implement that. Using a Hardware Description Language (HDL) called Gezel, the filter can be implemented on simulated hardware, thus making it obvious whether such implementation can serve the requirements.
	This report will discuss how the filter will be implemented through Gezel, as well as going more in depth on the topics: Modules of the CPU, Instruction set for the CPU, Integration of the CPU into a combined system, and finally, the critical parts of the CPU, i.e. speed, power consumption and size.\\
	For the purpose of this report, the data for the filter will be loaded through a file one data point at a time and then processed by the simulated hardware.\\
	
\subsection{Requirements}
Below follows a list of functional and non-functional requirements:\\

\textbf{ Functional requirements for the application:}
\begin{itemize}
	\item Each module of the processor must first be built, run and tested by their own
	\item An instruction set must be designed
	\item A CPU that can execute the instructions must be designed
	\item The CPU must be integrated into a larger system
	\item The value of the system must be analysed in terms of size, speed and power

\end{itemize}
\textbf{Non-functional requirements for the application:}
\begin {itemize}
	\item The system must be implemented with the use of Gezel
\end{itemize}

\section{Analysis}
 	In order to initiate the structure- and design-process of the program, a number of questions needed to be answered first:\\
 	
 	\begin{enumerate}
	\item How could the C-code be compiled into hardware instructions
	\item Which instructions would be needed to accomplish this?
	\item Which hardware modules would be needed to execute the instructions?	
	\item How could the CPU be designed in order to understand the instruction set?
	\item How could the CPU be integrated into a system?
	\item What would the critical parts of the system be and how could these be analysed?
\end{enumerate}

\subsection{Problem 1: Hardware modules}
	To gain a basic understanding of GEZEL and hardware design, an initial guide was given that suggested the creation of the following modules, as well as how to make these: Counter, Adder, Multiplexer, Arithmetic Logic Unit (ALU) with and without status flags, a register file and an instruction memory block. The main concern here was to understand the difference between software and hardware design.\\
	
\subsection{Problem 2: Instruction set}
	Designing the instruction set was a fairly arbitrary task, as there were very few limitations and the instructions in theory could be anything. The first question here was whether the instructions should be able to execute on single cycles or be more complex and execute across several cycles (RISC vs. CISC).\\
	The second question was how to ensure that these instructions could actually be interpreted by the processor, and what actual instructions needed to be there.\\
	\\
	
\subsection{Problem 3: Implementation of the CPU}
	Once the instruction set was finished, the next part would be to implement a CPU that could understand and execute the instruction set. To do this, a block diagram was to be designed to get an overview of how each component should be connected, but the design of this diagram would depend entirely on the instructions needed.\\
	Once the block diagram was completed and all components were connected, the next issue became to distinguish control- and data signals - here, control signals configure a single component whereas data signals carry data between the components.\\
	Finally, the last question was how to combine the buildingblocks from before into a functional CPU.\\
	
\subsection{Problem 4: System Integration}
	Having a functional CPU, the next step was to integrate it into a premade platform consisting of a bus and an external memory. To do this, the CPU needed to be able to communicate and react properly to the bus. The bus uses a master/slave protocol where the CPU is the master and the external memory the slave. The protocol between the master and the slave was predefined and functional in the provided Platform, 
	so the main objective was to implement the correct logic that would allow the master to send data to the slave, wait for a return signal as well as receive and save the returned data.\\
	
\subsection{Problem 5: Critical parts}
	There are three terms that needs to be discussed when it comes to the critical parts of a system: Size, speed and power.\\
	The speed of a dedicated system is the most important of these up to the point where you are certain that it is capable of making the maximum needed computations within the allowed time frame. In the case of this assignment it is a matter of handling 250 data points per second and display the result to the user. However, once this has been achieved, a further increase in speed will generally mean raising the unit cost as well as potential power consumption.\\
	Power and size are equally important, as the bigger the processor is, the more cumbersome the ECG scanner will be for a user to wear, and the more power it consumes, the more often it will need to be recharged or have the battery replaced. Therefore, the three problems becomes:
	
	\begin{itemize}
		\item Is the CPU fast enough and where is it mainly spending the time?
		\item Which modules are taking up space on the chip and what can be done to reduce this?
		\item How much energy does the system consume and what can be done to reduce that?
	\end{itemize}
	
\section{Implementation}
\subsection{Hardware modules}
	To build the initial modules, it was decided that each module should have a testbench associated in order to validate their functionality before progressing. The code for these modules as well as their testbenches can be seen in the appendix.\\
	The modules were then treated as basic building blocks for the CPU.\\
	
\subsection{Instruction set}
	For this report, RISC was chosen as a basis for the instruction set. A theoretical assembly code was made based on the assumption of single-clock instructions, where instructions were added to the list as they became needed. From this theoretical assembly code the instruction set was extracted.\\
The instructions were based on the idea that division by 30 could be approximated using multiplication of a constant along with bitshifting.
With this in mind, the instruction set ended up consisting of 11 separate instructions:\\
	
	\begin{itemize}
			
			\item LOAD R0 X: \qquad[$R0=mem(X)$]
			\item STRE X R0: \;\qquad[$mem(X)=R0$]
			\item MOVE R0 R1: \,\,\quad[$R0=R1$]
			\item ADD R0 R1 R2: \quad[$R0=R1+R2$]
			\item SUB R0 R1 R2: \quad[$R0=R1-R2$]
			\item ADDI R0 R1 X: \quad[$R0=R1+X$]
			\item SUBI R0 R1 X: \quad[$R0=R1-X$]
			\item SHRI R0 R1 X: \quad[$R0=R1>>X$]
			\item MULI R0 R1 R2: \quad[$R0=R1*R2$]
			\item BLILT R0 R1 X: \quad[$if\;(R0>R1)\;goto\;X$]
			\item BRCH X: \qquad\qquad[$goto\;X$]
		\end{itemize}
	The full assembly program can be seen in the appendix - in total, the instructions used are 17 lines long with the 11 different instructions written above.
\subsection{Implementation of the CPU}
	To get started with implementing the CPU, a block-diagram of the structure needed to be made. Looking at one instruction at a time, the diagram was drawn little by little, expanding and adding signals as they became needed. When the diagram appeared to include all the functionality needed, implementing the CPU was supposed to be a matter of using the previously made modules as building blocks, expanding a few of them with further signals and connecting it all as stated by the diagram. However, through this process many unexpected errors were encountered which was either due to quirks in GEZEL, misunderstandings on how hardware works or simple slips in the code that went unnoticed due to Gezels horrible error messages. Making sure all signals arrived at the correct destination at the desired time proved the hardest challenge.  In the end a working CPU was achieved and the diagram of this CPU can be seen below:\\
	\begin{figure}[h!]
		\centering
			\includegraphics[width=1\textwidth]{Screenshots/CPU diagram.png}
		\caption{The Design diagram of the CPU}
		\label{DesignDiagram}
	\end{figure}
	\\	
	
\subsection{System Integration}
	For attaching the CPU to an actual platform, a premade $Platform.fdl$ was supplied, containing a bus, an external memory as well as a template for a CPU datapath. The CPU template provided 3 output signals and 2 input signals. Communicating with the bus and external memory was then a matter of writing and reading the correct data on the signals at the right time. However, communicating with external devices in this way cannot be done on a single clock-cycle and will generally have to wait for a ready-signal. Therefore the original CPU diagram needed to be extended to support a stall signal. 
Furthermore, a few registers were added to the CPU datapath to allow for saving the control state while waiting for return signal, as well as maintaining the bus command until it returned.
	When the CPU had been updated and knew how to wait for a signal, a few more modifications were made to the overall diagram to pass data back and forth to the CPU external signals and finally the system was working.
	
\section{Critical parts}
	As was stated in the analysis, the critical parts involve three different factors to be measured: Speed, size and power.\\
	Speed is measured in number of clock cycles needed to execute a commands with a certain complexity related to the frequency. For this program there were 11 instructions. 7 of which takes 1 clock cycle to execute (Registers and Arithmetic), 2 that takes 2 cycles (jumps), and the last two that takes 8 and 9 clock cycles respectively (LOAD and STRE) - However, STRE is currently configured to only print data directly to the screen instead of saving in memory, which allows it to finish in 1 cycle instead of 9. For each data point after the 30'th, the program will carry out these instructions such that in total, each data point requires an estimated 27 clock cycles to evaluate. Furthermore, the processor needs to be able to handle 250 data points per second. With this knowledge, it's possible to calculate the minimum required frequency for the processor to run. By calculating the clocks pr. sample times the samplerate we get: $27*250=6750Hz=6.75kHz$. This is the minimum frequency, after the 30'th data sample. This may initially seem like a high amount of cycles, but when compared to the fact that a standard, all-purpose processor like the Core I5 (which is a mid-range processor introduced in 2009 by Intel and used in many computers nowadays) runs at 2.7 GHz for the lowest, which is equivalent to 2.700.000.000Hz, 6750Hz does not seem like much and should be easily accomplished by the processor developed here.\\
	In order analyse the potential maximum frequency for this system, it is important to realise that different commands don't necessarily take up an equal amount of time. The time needed to perform the calculation of each command is given by the critical path of the logic, meaning the longest path that a signal can take. The longer the signal needs to travel, the more time it will need to reach it's destination and the slower the clock will need to be, to make sure that everything has reached it's destination before the clock ticks again.
	In this way certain commands could run with a faster clock than others. However, it is not common practice to do constant scaling of the CPU frequency based on individual commands, which is why the frequency will need to be set to match the longest critical path. For instance: 1 multiplication followed by a division could take 10ns, whereas 2 additions and an \& might only take 6ns). Therefore, to get a better idea of this, it would be necessary to test each datapath for each command and determine which is the slowest. Given that the system is simulated by a HDL, running the code and measuring the time it takes cannot give any proper insight, as there are many more factors in play.\\
	The previously calculated frequency of 6.75kHz is based on the assumption that all commands take the same amount of time to execute. A guestimate on the optimal real frequency could therefore be around 15 KHz, which should take into account the discrepancy between different execution-times and still allow a little room for error\\
	\\
	In terms of size, the program has not been optimized in this way - first and foremost, the importance was to ensure the program ran correctly and that it ran fast enough to accomplish the necessary 250 data points per second. Many of the data signals have been implemented with excess amount of bits to be certain that the initial product would never run into an overflow. To optimize for size many of these signals could be analysed for maximum values and scaled down to more reasonable width.\\
	\\
	Finally, in terms of power, the main thing to look at would be the number of signals that switches value, also called toggles. Gezel has a built in option to analyse this, and a screenshot of this can be seen below, running through 250 data samples. (the full toggle report can be seen here: \ref{toggle_full}):\\
	\begin{figure}[h!]
		\centering
			\includegraphics[width=1\textwidth]{Screenshots/Screenshot_profiling.png}
		\caption{A screenshot of the output of the toggle profiling result.}
		\label{toggle}
	\end{figure}
	
	What can be seen on figure \ref{toggle} is the sum of toggles caused by the different low-level operations. This displays how many evaluations (Evals) and how many toggles there are for each 
	 list of cycles, their evaluations, the number of toggles and the values of all signals, be they 0 or 1. The interesting thing here is to look at the amount of toggles per cycle, as this is what can be optimized. To figure out where to optimize the program in terms of power, the whole simulation is profiled with toggling, the weakest points are identified and evaluated for smarter solutions.\\


\section{Design}
\subsection{Division by 30}
One of the more important decisions when designing the CPU was how to divide. This is not a simple thing to do in the hardware domain, and as a result, doing the division by 30 could cause problems. Based on new knowledge about hardware arithmetic it was clear that division is a much more complex operation than adding and subtracting, hence, it is beneficial to reduce number of divisions to a minimum. Initially this meant that dividing the whole sum by 30 instead of dividing each value by 30 and then summing, would improve the performance significantly.
Performing the actual division, however, proved a completely different challenge. Implementing division in hardware within few clock-cycles is by no means trivial. It was found that for this particular case, the level of precision did not matter as much as the difference between each sample, which is the only factor when finding peaks. Similarly, for measuring heartbeat intensity it was concluded that the value is arbitrary in either case. Finally, in the case of this particular assignment, all values used are integers, meaning that they are rounded off and not necessarily precise to begin with. For these reasons the decision for implementing division by 30 was instead to approximate it.
	Given that multiplication is much simpler to implement than division and that bitshifting is close to free, a python script \ref{div30approx} was made to calculate the best approximation to $1/30$ by combining multiplication and bitshifting, while staying within 32-bit values.
	It was found that a reasonable approximation could be made by multiplying by $4369$ and bitshifting right by $17$. By changing the original C-code to perform this approximation instead it was possible to compare the results to the original values. On a dataset of 10.000 samples this comparison yielded a difference on only 374 of those sample, with a maximum error of 1 (possibly less as it was comparing integers). This was found to be acceptable, as the values the filter operates on are at times in the several thousands, meaning that the approximation are off by less than 1\textperthousand. Furthermore, it should be noted that the provided data is already rounded off by the fact that they are integers.\\

\subsection{Memory access}
Accessing external memory also proved to be the slowest of all commands, spending up to 9 cycles doing it, whereas handling registers only take 1 cycle. Going back and looking at the original C-implementation from Assignment 1 made it obvious that it could be optimized greatly. Instead of dividing and summing the past 30 values for each sample, the sum was implemented as an accumulator, which would maintain the sum of the past 30 value across clock-cycles and only subtract the oldest value as well as adding the latest value, thus moving the sum instead of redoing it. By subsequently multiplying and bitshifting the whole sum, the number of instructions per sample would go down from 30 additions and 30 divisions to a single addition, a subtraction, a multiplication and a bitshift. This would cause a dramatical increase in calculation speed. To let the filter start as soon as the CPU powers on, it was further more decided to use 0-values for the subtraction until the algorithm would arrive at the 30th sample, after which it would use the 30-samples old values. This meant that the output from the filter would grow slowly in intensity until having gathered the initial 30 samples, thus making the data invalid prior to the 30th sample.
	
\subsection{Decoding instructions}
	To ensure the CPU could access and understand the instruction set, the instructions needed a certain format. From using Gezel for the development along with implementing the instruction memory using an ipblock in Gezel, the instructions needed to be listed in a textfile and formatted as hexadecimal numbers, each command having a unique address and a command separated by a space. The commands were put in a file called program.txt, which would allow the ipblock in Gezel to open it and access the different addresses similar to an actual memory block.
	While implementing the different instructions it turned out that changing hex numbers, knowing what is where and testing individual cases proved difficult. For this reason an actual assembler was written in Python \ref{assembler}, which would take the assembler commands and write out the program.txt file in a format that Gezel would then understand. This sped up the development significantly, reduced human error as well as made it possible to program the CPU directly by assembly commands.
		For the sake of clarity the different instructions were furthermore implemented in the controller through a Finite State Machine with only one state. This allowed for execution of separate Signal Flow Graphs for the individual instructions, grouping their signals together and easing the development and debugging. A pure combinatorial implementation could have been used instead but was found to clutter the overview and make it harder to figure out exactly what was going on for each instruction.\\
	As it turned out that the ipblock used for instruction memory contained a register on the inside, there was no direct access to the current address read. Therefore the system needed to stay one command behind all the time. 
	Furthermore, given the fact that the ipblock could only be changed with a 1-clock delay, implementing a jump-signal required support for a built-in stall-signal for the system to wait till the ipblock could return the correct address. This was fixed by adding a stall-signal from the controller and setting it high for each jump-command which would stall for 1 cycle, effectively adding a NOP and letting the ipblock catch up.
	
\section{Improvements}
To improve on the design, changing the value of the MWI filter to sum over and divide by 32 would make both the implementation, maintenance and all the calculations involved much simpler. This is true for any value of $2^{x}$ where $x$ is a whole number. Dividing by 32 is a matter of bitshifting right by 5 and no further calculations is necessary. Doing this would not include more additions either, thanks to the way that the summation has been implemented - all that should be changed would be to change the reference of how many steps back the program should go before subtracting. Another benefit from using 32 instead of 30 would be to simplify implementation of a circular buffer. In assignment 1 a circular buffer was implemented by a struct, a method for incrementing the counter and a method for resetting the counter, when it should loop. Making a circular buffer of 32 values, the counter only needs be 5 bits thus making the reset method redundant as the counter would automatically reset on overflow.
Using 32 instead of 30 in this case would extend the moving window ever so slightly thus causing a tiny increase in the filter group delay. However, given that this is of very little importance in our particular case, it was concluded that the choice must have been for educational reasons only.
\\
A further improvement could be to combine the addition and the subtraction into a single accumulator command which could run on a single cycle. This would spare one clock cycle data sample and due to addition and subtraction being simple operations, combining these would probably not extend the critical path of the system. Similarly multiply and bitshift can be combined into a single "divide by 30 approx" command. Even though multiply is a fairly complex operation, in this case it will always multiply by a constant of 4369 which simplifies the operation. Bitshifting on the other hand, is completely free as it can be a case of connecting certain wire-ends on one side to other wire-ends on another. Therefore combining multiply and bitshift into 1 command would probably not extend the critical path either and thus save the system another 1 cycle per datasample.\\
\\
The final improvement that this report has been able to identify would be to optimize the signal width; according to the data used in this report, no point is larger than 10.000, which means that no signal needs be larger than 14 bit, and thus could be reduced quite a lot - to take into account, they could be reduced to a bitwidth of about 16 and still be within the limits of the data provided.\\

\section{Conclusion}
It has now been successfully demonstrated that it is possible to implement a dedicated processor for the task of calculating the data of the Mowing Window Integration filter. Furthermore, the critical parts of the processor has been analysed, and the speed of the processor has been determined, along with a way to optimize it in terms of both size and power.
\newpage
\begin{thebibliography}{9}

\bibitem{lamport94}
  Michael Reibel Boesen, Jan Madsen, Linas Kaminskas, Paul Pop, Karsten Juul Frederiksen\\
  \emph{Assignment 2: The ECG processor}\\
  2013.\\

\bibitem{Gezel}
  \emph{Lecture7: Finite state machine with Datapath}\\
  Fall, 2007.\\

\bibitem{GezelBasicSyntax}
  \emph{GEZEL Basic Syntax}\\
  
\bibitem{ClockSpeeds}
  \emph{http://smallbusiness.chron.com/ghz-mean-computer-processor-66857.html}
  Used to explain clock speeds of a processor
  
\bibitem{CoreI5}
  \emph{http://www.intel.com/content/www/us/en/processors/core/core-i5-processor.html}
  Used to determine clock speeds of Core I5
\end{thebibliography}
	
\newpage	
	\begin{Large}
		\textbf{Appendix}
	\end{Large}
	\appendix

\section{Who wrote what}
\underline{Jacob Gjerstrup, s113440 wrote:}\\
\textbf{Report:} Abstract, Introduction, Analysis, Implementation (Hardware Modules, Instruction set), Conclusion. Critical Parts and Design were made in collaboration with Jakob \\
\textbf{Code:} Hardware modules (adder.fdl, ALU.fdl, ALU-withFlags.fdl, counter.fdl, instructionMemory.fdl, mux.fdl, register.fdl, testbech.fdl). Assembler.py was made in collaboration with Jakob\\
\\
\underline{Jakob Welner, s124305 wrote:} \\
\textbf{Report:} Implementation (Implementation of the CPU, System Integration). Critical Parts and Design were made in collaboration with Jacob \\
\textbf{Code:} CPU.fdl, div30approx.py and Platform.fdl. Assembler.py was made in collaboration with Jacob\\

\section{Sourcecode}

\subsection{Source code for the Processer modules}
	\subsubsection{Counter}
		\lstinputlisting[language=C]{Code/Proc_Mods/counter.fdl}	
	\subsubsection{Adder with testbench}
		\lstinputlisting[language=C]{Code/Proc_Mods/adder.fdl}	
	\subsubsection{Multiplexer}
		\lstinputlisting[language=C]{Code/Proc_Mods/mux.fdl}	
	\subsubsection{Arithmetic Logic Unit (ALU)}
		\lstinputlisting[language=C]{Code/Proc_Mods/ALU.fdl}
	\subsubsection{ALU with flags}
		\lstinputlisting[language=C]{Code/Proc_Mods/ALU_withFlags.fdl}	
	\subsubsection{Register}
		\lstinputlisting[language=C]{Code/Proc_Mods/register.fdl}	
	\subsubsection{Instruction Memory}
		\lstinputlisting[language=C]{Code/Proc_Mods/memory.fdl}	
\subsection{Instruction set}
	\lstinputlisting[language=C]{Code/assembly_mwint.asm}	
\subsection{CPU}
	\lstinputlisting[language=C]{Code/CPU.fdl}
\subsection{Profiling}
	\label{toggle_full}	
	\lstinputlisting[language=C]{Code/toggle_output.txt}

\subsection{Python scripts}
	The script to find the division approximation:
	\label{div30approx}
	\lstinputlisting[language=Python]{Code/div30Approx.py}
	
	The script to make the assembly OP-codes:
	\label{assembler}
	\lstinputlisting[language=Python]{Code/assembler.py}
\end{document}